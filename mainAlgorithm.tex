\documentclass{article}
\usepackage{graphicx} % Required for inserting images

\usepackage[a4paper, margin=2.5cm]{geometry}

\usepackage[leqno,intlimits]{amsmath}
%%%%% This is important, as it contains the basic math stuff. leqno means equation numbers are on the left, not on the right. intlimits puts limits of integrals in top/bottom as opposed to next to the integral sign.


\usepackage{amsmath, amssymb, amsthm}
\usepackage{graphicx}
\usepackage{booktabs}
\usepackage{caption}
\usepackage{hyperref}
\usepackage{float}

\usepackage{siunitx}

\usepackage{amssymb}
%%%%% This one is for math symbols like \mathbb R
\usepackage{bbm}
%%%%% For indicator function as \mathbbm{1}
\usepackage{amsthm}
%%%%% Theorem, lemma, proof etc. environments
\usepackage{hyperref}
%%%%% This one gives clickable links for equations and references in the pdf file.

\usepackage{showkeys} %***
%%%%% This one is very useful: it shows you in the dvi/pdf the codes for your references; try it after you have some equation labels (uncomment it)!

%\usepackage{refcheck}
%%%%% Need to check this, from P Mora.
%\usepackage{color}
%\usepackage{soul}
%\setstcolor{red}
%%%%% With these, you can highlight text like a marker, or cross things out. Look up the manual of the soul package on the Web.

%%%\usepackage{pst-all}
%%%\usepackage{pstricks-add}
%%%%% These are extensions to pstricks, see below

%\usepackage{pstricks}
%\usepackage{graphicx}
%%\usepackage{multido}
%%\usepackage{calc}
%%%%% These are the ones I use for drawing pictures. Look up pstricks in google, then --> documentation. Don't need this for first tries with LaTeX.

\usepackage{enumerate}
\usepackage{enumitem}
%%%%% With these you can let the enumerate environment know what type of lists you prefer (eg. (a), (b), (c); or maybe [i], [ii], [iii], [iv], etc.)

%\usepackage{array}
%%%%% This one contains nice extensions to the tabular environments (that is, tables).

%%%\usepackage{pifont}
%%%%% Usage: \ding{number}, funny symbols in addition to standard LaTeX ones.

%\usepackage[super]{nth}
%\nth{1}, \nth{2}, \nth{3}, \nth{4} for 1st, 2nd, 3rd, 4th, etc

%\pagestyle{empty}
%%%%% This makes LaTeX forget about headers, page numbers, etc. Comment it out if you need page numbers.

%\psset{unit=1pt}
%\psset{linewidth=0.5pt}
%%%%% This sets units for drawing with pstricks, see above.

\newtheorem{tm}{Theorem}[section]
\newtheorem{lm}[tm]{Lemma}
\newtheorem{pr}[tm]{Proposition}
%%\newtheorem{rem}[tm]{Remark}
%%%%% If you need theorem-like environments.

\numberwithin{equation}{section}
%%%%% This gives you nicer equation numbering if you have sections in an article.

%\newcommand*{\hop}{\bigskip\noindent}
%%%%% Skips without indenting

\newcommand*{\e}[1]{\text{\rm e}^{#1}}
%%%%% This is how I make exponentials; usage: $\e{-x}$

%\newcommand*{\di}{\,\text{d}}
%%%%% This is what I use for "d" in dx in integrals.

%\newcommand*{\dd}[1]{\frac{\text{d}}{\text{d}{#1}}}
%\newcommand*{\pp}[1]{\frac\partial{\partial{#1}}}
%%%%% And these for derivatives.

%\newcommand*{\Om}{\Omega}
%\newcommand*{\om}{\omega}
%\newcommand*{\vp}{\varphi}
%\newcommand*{\vr}{\varrho}
%\newcommand*{\te}{\theta}
%\newcommand*{\la}{\lambda}
%\newcommand*{\ve}{\varepsilon}
%\newcommand*{\de}{\delta}
%\newcommand*{\al}{\alpha}
%\newcommand*{\si}{\sigma}
%\newcommand*{\pt}{\partial}

%\newcommand*{\Rb}{\mathbb R}
%\newcommand*{\Cb}{\mathbb C}
%\newcommand*{\Nb}{\mathbb N}
%\newcommand*{\1b}{\mathbbm1}
%\newcommand*{\Fc}{\mathcal F}
%\newcommand*{\Pc}{\mathcal P}
%\newcommand*{\Nc}{\mathcal N}
%\newcommand*{\ordo}{\mathfrak o}
%\newcommand*{\Ordo}{\mathcal O}

%\newcommand*{\Pv}{{\bf P}}
%\newcommand*{\Ev}{{\bf E}}
%\newcommand*{\Vv}{{\text{\bf Var}}}
%\newcommand*{\Cov}{{\text{\bf Cov}}}
%\newcommand*{\Corr}{{\text{\bf Corr}}}

%\newcommand*{\wih}{\widehat}
%\newcommand*{\wt}{\widetilde}
%\newcommand*{\un}[1]{\underline{#1}}

%%\renewcommand\tabularxcolumn[1]{>{\centering\arraybackslash}m{#1}}
%%%%% This is some advanced tabular stuff I find useful with the tabularx environment. Never mind it.
\usepackage{tikz}
\usepackage{amsmath}
\usepackage{circuitikz}
\usepackage{float}
\usepackage{enumitem}
\usepackage{graphicx} 
\usepackage{subfigure} 
\usepackage{bbm}
\usepackage{amsmath, amssymb}
\usepackage{geometry}
\usepackage{graphicx}
\usepackage{booktabs}
\usepackage{caption}
\usepackage{hyperref}
\usepackage{float}
\usetikzlibrary{automata, positioning}

\begin{document}

\title{Math model in public opinion classification}
\author{}
\maketitle
\thispagestyle{empty}
\clearpage

\maketitle
\tableofcontents
\thispagestyle{empty}
\newpage
\setcounter{page}{1}
\clearpage

\maketitle

\section*{Random variables}

\begin{itemize}

    \item  Given the set of arbitrary opinion categories $X$, the indicator function of the subset $C$ of $X$ is 
    \[
    C_i = \mathbf{1}_{\{\text{object} \in \text{class } i\}},
    \]
where $ C_i \in \{0, 1\} $, and $ \mathbf{1}_{\{\cdot\}} $ denotes the indicator function;

    \item $ P_i $ is defined as the priority weight of the category $ i $, where $ P_i \in [0,1]$;
    
    \item $ S_i $ is defined as the sentiment polarity threshold for public opinion of category $ i $
    \[
    S_i =
    \begin{cases}
    \in [-1,0), & \text{negative sentiment }, \\
    \in (0,1], & \text{positive sentiment}.
    \end{cases}
    \]
    
    \item $ R_i $ is defined as the degree of relevance between the public opinion and category $ i $, where $ R_i \in [0,1] $.

\end{itemize}

\section*{Model Parameters}

\begin{itemize}

    \item $ k $ is the negative sentiment amplification coefficient ($ k > 1 $), used to amplify the impact of negative emotions;
    
    \item $ w $ is the priority decay coefficient ($ 0 < w < 1 $), used to reduce the influence of secondary categories;
    
    \item $ m $ is the maximum number of categories considered ($ m = 2 $).
\end{itemize}

\subsection*{3.1 Risk Intensity Model}

Define the risk intensity $ R_i $ of category $ i $ given sentiment polarity threshold $ S_i $:

\[
R_i = 
\begin{cases} 
0, & \text{if } S_i \geq \theta_i, \\
k \cdot \dfrac{\theta_i - S_i}{1 + \theta_i}, & \text{if } S_i < \theta_i.
\end{cases}
\]

\begin{itemize}
    \item $k$ is risk amplification coefficient, emphasising the destructive impact of negative sentiment where $ k = 1.2 $; 
    \item Denominator $ 1 + \theta_i $: Normalization factor ensuring $ R_i \in [0,1] $;
    \item When $ S_i = -1 $ (extreme negative), $ R_i = k \cdot \dfrac{\theta_i + 1}{1 + \theta_i} = k \approx 1.2 $, but constrained by subsequent formulas to remain in $[0,1]$.
\end{itemize}


\subsection*{3.2 Comprehensive Emergency Severity Model}

\textbf{Case 1: Single-category incident (category $ i $)}
\[
U = P_i \cdot R_i
\]

\textbf{Case 2: Dual-category incident (primary category $ i $, secondary category $ j $)}
\[
U = 1 - (1 - P_i \cdot R_i) \cdot (1 - P_j \cdot R_j)^{\delta}
\]

\textbf{Parameter Explanation:}
\begin{itemize}
    \item $ \delta = 0.65 $: Secondary category attenuation coefficient ($ 0 < \delta \leq 1 $);
    \item When $ \delta = 1 $, risks are fully independent;
    \item When $ \delta < 1 $, the influence of the secondary category is attenuated.
\end{itemize}






\section*{III. Business-Probability Mapping Framework}

\subsection*{3.1 Optimized Risk Intensity Model}

\textbf{Dynamic Threshold Adjustment:}
$$
R_i = 
\begin{cases} 
0, & \text{if } S_i \geq \theta_i(t) \\
k \cdot \dfrac{\theta_i(t) - S_i}{1 + \theta_i(t)} \cdot \alpha_{ij}, & \text{if } S_i < \theta_i(t)
\end{cases}
$$

where:
\begin{itemize}
    \item $ \theta_i(t) = \theta_i(0) + \beta \cdot \dfrac{N_{\text{high-risk}}(t)}{N_{\text{total}}(t)} $: dynamic threshold;
    \item $ \alpha_{ij} = 1 + \gamma \cdot P(C_j \mid C_i) $: risk propagation enhancement factor;
    \item $ \beta = 0.1 $: learning rate;
    \item $ \gamma = 0.3 $: propagation strength coefficient.
\end{itemize}

\textbf{Uncertainty Quantification:}
$$
R_i \sim \mathcal{N}\left(\mu_i, \sigma_i^2\right)
$$
where:
\begin{itemize}
    \item $ \mu_i = k \cdot \mathbb{E}\left[\dfrac{\theta_i(t) - S_i}{1 + \theta_i(t)} \cdot \mathbb{I}(S_i < \theta_i(t))\right] $;
    \item $ \sigma_i^2 = \text{Var}\left(\dfrac{\theta_i(t) - S_i}{1 + \theta_i(t)} \cdot \mathbb{I}(S_i < \theta_i(t))\right) $.
\end{itemize}

\textbf{Interpretation:}
\begin{itemize}
    \item Dynamic threshold $ \theta_i(t) $: automatically adjusts based on historical data; increases by 20\% during major events;
    \item Risk propagation coefficient $ \alpha_{ij} $: amplifies risk when a propagation path from $ i $ to $ j $ exists;
    \item Uncertainty quantification: triggers manual review when $ \sigma_i > 0.1 $.
\end{itemize}

\subsection*{3.2 Optimized Comprehensive Emergency Severity Model}

\textbf{Optimized Model: Bayesian Network Enhanced Framework}
$$
U = 1 - \prod_{k \in \mathcal{C}} (1 - P_k \cdot R_k)^{\omega_k}
$$
where:
\begin{itemize}
    \item $ \mathcal{C} $: set of relevant categories (at most two most relevant categories);
    \item $ \omega_k = \dfrac{P(C_k \mid \mathcal{F}_t)}{\sum_{j \in \mathcal{C}} P(C_j \mid \mathcal{F}_t)} $: Bayesian weight;
    \item $ P(C_k \mid \mathcal{F}_t) $: posterior category probability given current information set $ \mathcal{F}_t $.
\end{itemize}

\textbf{Risk Propagation Path Consideration:}
$$
P(C_j \mid \mathcal{F}_t) = \sum_{i \in \mathcal{C}} P(C_j \mid C_i) \cdot P(C_i \mid \mathcal{F}_{t-1})
$$

\textbf{Uncertainty Propagation:}
$$
U \sim \mathcal{N}\left(\mu_U, \sigma_U^2\right), \quad \text{where } \sigma_U^2 = \sum_{k \in \mathcal{C}} \omega_k^2 \cdot \sigma_k^2
$$

\noindent\textbf{Rule:} When $ \sigma_U > 0.08 $, manual review is triggered regardless of $ U $ value.

\subsection*{3.4 Probabilistic Interpretation of Typical Scenarios}

\textbf{Scenario 1: High-risk Ideological + Medium-risk Regional}
\begin{itemize}
    \item Parameters: $ P_1 = 1.0, R_1 = 0.95 $; $ P_7 = 0.92, R_7 = 0.60 $
    \item Risk propagation: $ P(7 \mid 1) = 0.23 $
    \item Calculation:

    \begin{align*}
    \omega_1 &= \dfrac{P(C_1 \mid \mathcal{F}_t)}{P(C_1 \mid \mathcal{F}_t) + P(C_7 \mid \mathcal{F}_t)} = \dfrac{0.95}{0.95 + 0.23 \times 0.60} = 0.87 \\
    \omega_7 &= 1 - \omega_1 = 0.13 \\
    U &= 1 - (1 - 1.0 \times 0.95)^{0.87} \times (1 - 0.92 \times 0.60)^{0.13} \\
    &= 1 - (0.05)^{0.87} \times (0.448)^{0.13} \\
    &= 0.96
    \end{align*}
    
    \item \text{Probability Interpretation: Even with medium regional risk, high ideological risk and propagation path make overall risk nearly certain.}
    \item \text{Business Conclusion: Mark as Red; report within 24 hours.}
\end{itemize}

\textbf{Scenario 2: Medium-risk Cadre + Low-risk Talent (with uncertainty)}
\begin{itemize}
    \item Parameters: $ P_2 = 0.85, R_2 \sim \mathcal{N}(0.60, 0.05^2) $; $ P_4 = 0.62, R_4 \sim \mathcal{N}(0.30, 0.08^2) $
    \item Uncertainty propagation:
    
    \begin{align*}
    \sigma_U^2 &= \omega_2^2 \cdot \sigma_2^2 + \omega_4^2 \cdot \sigma_4^2 \\
    &= 0.72^2 \times 0.05^2 + 0.28^2 \times 0.08^2 = 0.0021 \\
    \sigma_U &= 0.046 < 0.08 \quad \text{(no manual review needed)}
    \end{align*}

    \item \text{Calculation:}

    \begin{align*}
    U &= 1 - (1 - 0.85 \times 0.60)^{0.72} \times (1 - 0.62 \times 0.30)^{0.28} \\
    &= 1 - (0.49)^{0.72} \times (0.814)^{0.28} = 0.63
    \end{align*}
    
    \item Probability Interpretation: Combined risk exceeds yellow threshold, but uncertainty is low.
    \item Conclusion: Mark as Yellow; verify within 72 hours.
\end{itemize}

\textbf{Scenario 3: High-risk Institutional Execution (Single Category, High Uncertainty)}
\begin{itemize}
    \item Parameters: $ P_5 = 0.55, R_5 \sim \mathcal{N}(0.90, 0.12^2) $
    \item Uncertainty analysis: $ \sigma_U = \omega_5 \cdot \sigma_5 = 1.0 \times 0.12 = 0.12 > 0.08 $
    \item Calculation: $ U = 0.55 \times 0.90 = 0.50 $
    \item Probability Interpretation: High sentiment polarity but low category priority and high uncertainty.
    \item Conclusion: Mark as Yellow \textbf{and} trigger manual review (despite $ U = 0.50 $).
\end{itemize}

\section*{IV. Model Optimisation Directions and Probabilistic Recommendations}

\subsection*{4.1 Dynamic Threshold Adjustment Mechanism}

\textbf{Bayesian Dynamic Threshold}
$$
\theta_i(t) = \theta_i(0) + \beta \cdot \dfrac{N_{\text{high-risk}}(t)}{N_{\text{total}}(t)} + \eta \cdot E_t
$$
where:
\begin{itemize}
    \item $ \beta = 0.1 $: base learning rate;
    \item $ \eta = 0.05 $: major event enhancement coefficient;
    \item $ E_t = \begin{cases} 1, & \text{during major events} \\ 0, & \text{otherwise} \end{cases} $
\end{itemize}

\subsection*{4.2 Risk Propagation Network Model}

\textbf{Bayesian Network}
\begin{itemize}
    \item \textbf{Nodes:} 7 public opinion categories;
    \item \textbf{Edge weights:} $ P(C_j \mid C_i) $ estimated from historical data;
    \item \textbf{Inference:} Use belief propagation to compute posterior probabilities.
\end{itemize}

\textbf{Transition Probability Matrix Example:}
\begin{table}[H]
\centering
\begin{tabular}{lcccccc}
\toprule
From & Ideological & Regional & Cadre & ... \\
\midrule
Ideological & 0.62 & 0.23 & 0.08 & ... \\
Regional & 0.15 & 0.58 & 0.12 & ... \\
Cadre & 0.18 & 0.11 & 0.55 & ... \\
\bottomrule
\end{tabular}
\end{table}

\textbf{Risk Propagation Prediction:}
$$
P(C_j^{(t+n)} \mid \mathcal{F}_t) = \sum_{i=1}^7 P(C_j^{(t+n)} \mid C_i^{(t)}) \cdot P(C_i^{(t)} \mid \mathcal{F}_t)
$$

\textbf{Advantages:}
\begin{itemize}
    \item Early warning: predict risk escalation 1.5–2.5 days in advance;
    \item Resource optimization: focus monitoring on high-propagation paths;
    \item Precise intervention: disrupt key propagation channels.
\end{itemize}

\subsection*{4.3 Uncertainty Quantification}

\textbf{Uncertainty Propagation}
$$
\sigma_U^2 = \sum_{k \in \mathcal{C}} \omega_k^2 \cdot \sigma_k^2
$$

\textbf{Manual Review Trigger:}
$$
\text{if } \sigma_U > \tau \quad \text{then trigger human review}, \quad \tau = 0.08
$$


\textbf{Integrated Uncertainty:}
$$
\sigma_k^2 = \alpha \cdot \sigma_{SA}^2 + \beta \cdot \sigma_{CL}^2 + \gamma \cdot \sigma_{DS}^2
$$

\section{$\theta_f$}



\end{document}


$P_i$
$S_i$
$R_i$
Parameters: 
$k$ 
$w$ 
$m$

\bibliographystyle{}
\bibliography{}



\end{document}




