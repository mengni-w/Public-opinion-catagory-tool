\documentclass[11pt,a4paper]{article}
\usepackage[UTF8]{ctex} % 中文支持
\usepackage{amsmath, amssymb, amsthm}
\usepackage{graphicx}
\usepackage{booktabs}
\usepackage{longtable}
\usepackage{geometry}
\usepackage{fancyhdr}
\usepackage{hyperref}
\usepackage{enumitem}
\usepackage{caption}
\usepackage{lipsum} % 仅用于占位,可删除
\usepackage[utf8]{inputenc}
\usepackage[T1]{fontenc}
\usepackage[english, chinese]{babel}
\usepackage{booktabs}
\usepackage{siunitx}
\usepackage{ctex} % 优先使用 ctex 支持中文
\usepackage{float}
\usepackage{rotating}   % 关键:用于旋转表格
\usepackage{tabularx} 
\usepackage[utf8]{inputenc}
\usepackage[T1]{fontenc}
\usepackage[english, chinese]{babel}
\usepackage{xcolor}

% 页面边距设置
\geometry{left=2.5cm, right=2.5cm, top=2.5cm, bottom=2.5cm}

% 页眉页脚设置
\pagestyle{fancy}
\fancyhf{}
\rfoot{\thepage}
\renewcommand{\headrulewidth}{0.4pt}
\renewcommand{\footrulewidth}{0pt}

% 标题格式
\title{\textbf{贝叶斯与Dempster-Shafer证据理论的舆情风险评估模型}}
\author{吴梦妮}
\date{2025年10月15日}

% 定义关键词环境
\newenvironment{keywords}{
    \par\vspace{1em}\noindent\textbf{关键词:}%
}{\par}

% 定义案例环境
\newenvironment{example}[1]{%
    \par\medskip\noindent\textbf{案例#1:}%
}{\par\medskip}

% 调整列表间距
\setlist[itemize]{topsep=0pt, itemsep=0pt, partopsep=0pt, parsep=0pt}
\setlist[enumerate]{topsep=0pt, itemsep=0pt, partopsep=0pt, parsep=0pt}

% 超链接设置(可选)
\hypersetup{
    colorlinks=true,
    linkcolor=black,
    filecolor=magenta,      
    urlcolor=cyan,
    pdftitle={舆情风险智能评估模型},
    pdfauthor={舆情智能分析研究组},
}

\begin{document}

\maketitle

\begin{abstract}
在舆情风险领域,评估面临两大核心挑战:\\

数据稀缺:敏感舆情历史记录难以系统采集,传统基于大数据的机器学习模型难以部署;\\

判断模糊:风险判定需区分“正常批评”与“政治攻击”,但边界模糊,依赖经验,主观性强。\\

现有模型多采用固定阈值或线性加权,忽视了证据的不确定性与可靠性差异,易导致误报或漏报。\\
本文提出一种高阶舆情风险评估模型,融合贝叶斯决策理论、差异显著性检验与Dempster-Shafer(D-S)证据融合机制,构建“证据评估—不确定性推理—最优决策”三位一体的数学框架。\\

其亮点在于:\\

将差异显著性检验(p value)引入舆情评估,用于动态评估知识库匹配的统计代表性,而非简单比较数值;\\

在舆情领域应用D-S理论,输出信念区间(Bel/Pl),实现区间决策,而非伪概率;\\

将损失函数、统计检验、证据融合三者统一于贝叶斯决策框架,形成闭环逻辑;\\

本模型不检验 $S_i$ 与 $θ_k$是否相等(二者维度不同),而是检验:“当前舆情的情感极性 $S_i$ 是否显著偏离知识库所代表的‘典型高风险舆情’分布”。\\
\end{abstract}



\begin{keywords}
舆情风险评估;贝叶斯决策;Dempster-Shafer理论;差异显著性检验;证据融合;不确定性推理
\end{keywords}

\clearpage

\section{Background knowledge}

\subsection{Bayes' Theorem}
模型的基础是贝叶斯定理:
\begin{equation}
P(\theta|x) = \frac{P(x|\theta)P(\theta)}{P(x)}
\end{equation}
where:
\begin{itemize}[leftmargin=*]

    \item $P(\theta|x)$ is the posterior probability: the probability of parameter $\theta$ given observation $x$;
    \item $P(x|\theta)$ is the likelihood: the probability of observing $x$ given parameter $\theta$;
    \item $P(\theta)$ is the prior probability: the initial belief about $\theta$ before observing $x$;
    \item $P(x)$ is the marginal likelihood: the total probability of observing $x$ under all possible hypotheses, $P(x) = \sum_{\theta \in \Theta} P(x|\theta)P(\theta)$.
\end{itemize}

\subsubsection{决策空间与损失函数}

\subsubsection{设定参数:}
\begin{itemize}[leftmargin=*]
    \item \textbf{状态空间} $\Theta$是可能的真实状态集合。$\Theta = \{\theta_0, \theta_1\}$,其中 $\theta_0$表示舆情是低风险,$\theta_1$表示舆情是高风险;
    \item \textbf{决策空间} $\mathcal{A}$是可能的决策集合。$\mathcal{A} = \{a_0, a_1\}$,其中 $a_0$表示判定低风险,$a_1$表示判定高风险;
    \item $L(a,\theta)$ 是损失函数,量化在真实状态为 $\theta$ 时采取动作 $a$ 的成本;\\
    \textbf{损失函数}:
\begin{itemize}[leftmargin=*]
    \item $C_{\text{miss}} = L(a_0, \theta_1)$:漏报损失(判定低风险but实际高风险)
    \item $C_{\text{false}} = L(a_1, \theta_0)$:误报损失(判定高风险but实际低风险)
\end{itemize}
    \item $S_i \in [-1, 0]$:来自情感分析模型判断出的实际情感极性值,表示整体情感极性,数值越小情感越负面;
    \item $c_k \in [-0.95, -0.55]$:知识库匹配阈值;
     \item  $c_m$:是自适应风险建模模型,依赖外部知识,比如主体,行业,历史数据,政策背景,时间,发布者等因素,动态生成“决策标准”;
    \item $\theta_f$:最终自适应阈值,用于精准风险评估
\end{itemize}

\subsubsection{最优决策规则}

给定观测值 $S_i$,基于最小化期望损失,最优决策为:
\begin{equation}
\delta^*(S_i) = \arg\min_{a \in \mathcal{A}} r(a \mid S_i) = \arg\min_{a \in \mathcal{A}} \sum_{\theta \in \Theta} L(a, \theta) P(\theta \mid S_i) \tag{2}
\end{equation}
其等价于:
\begin{equation}
\text{选择 } a_1 \iff P(\theta_1 \mid S_i) \geq \tau, \quad \text{其中 } \tau = \dfrac{C_{\text{false}}}{C_{\text{false}} + C_{\text{miss}}} \tag{3}
\end{equation}

\textbf{阈值形式}:

\begin{equation}
\text{选择} a_1 \iff \frac{P(x|\theta_1)}{P(x|\theta_0)} \geq \frac{C_{\text{false}}P(\theta_0)}{C_{\text{miss}}P(\theta_1)} = \gamma, \quad \text{其中$\gamma$ 是似然比阈值}
\end{equation}
\\

最优决策阈值由\textbf{损失比} $\dfrac{C_\text{false}}{C_{\text{miss}}}$ 和\textbf{先验概率比} $\dfrac{P(\theta_0)}{P(\theta_1)}$ 共同决定。

\subsection{差异显著性检验:评估知识库的统计代表性}
本模型检验:"当前舆情的情感极性 $S_i$ 是否显著偏离知识库所代表的'典型高风险舆情'分布"。

\subsubsection{模型设定}
\begin{itemize}
    \item 设知识库代表一个历史高风险舆情的\textbf{分布}:$S \sim \mathcal{N}(\mu_k, \sigma_k^2)$,其中:
    \begin{itemize}
        \item $\mu_k$:历史高风险舆情的平均情感极性(如意识形态类 $\mu_k = -0.80$);
        \item $\sigma_k$:该分布的标准差;
    \end{itemize}
    \item $\theta_k$:\textbf{决策阈值},由损失函数与分布参数推导得出(见第4.2节),\textbf{非输入参数}。
    \item 观测值:$S_i$,当前舆情的情感极性。
\end{itemize}

\subsubsection{假设检验}
\begin{itemize}
    \item $H_0$:$S_i$ 来自知识库代表的高风险分布,即 $S_i \sim \mathcal{N}(\mu_k, \sigma_k^2)$;
    \item $H_1$:$S_i$ 来自其他分布(可能为误匹配、新风险、噪声)。
\end{itemize}

\textbf{检验统计量}:
\begin{equation}
z_k = \dfrac{S_i - \mu_k}{\sigma_k} \tag{4}
\end{equation}

\textbf{双侧p value}:
\begin{equation}
p_k = 2 \cdot \left(1 - \Phi(|z_k|)\right) \tag{5}
\end{equation}
其中 $\Phi(\cdot)$ 为标准正态累积分布函数。

\subsubsection{业务解释与置信度调整}
p value衡量:\textbf{在知识库代表的分布下,观测到当前 $S_i$ 的可能性}。
\begin{itemize}
    \item $p_k \leq 0.01$:极显著偏离 $\rightarrow$ 知识库\textbf{不适用} $\rightarrow$ 降低知识库置信度;
    \item $0.01 < p_k \leq 0.05$:显著偏离 $\rightarrow$ 知识库\textbf{可能过时} $\rightarrow$ 适度降权;
    \item $0.05 < p_k \leq 0.10$:轻微偏离 $\rightarrow$ 知识库\textbf{基本适用};
    \item $p_k > 0.10$:不显著 $\rightarrow$ 知识库\textbf{可靠}。
\end{itemize}

\textbf{置信度调整函数}:
\begin{equation}
c_k^{\text{adj}} = c_k \cdot \max\left(0.3, \, 1 - \dfrac{p_k}{0.1}\right) \tag{6}
\end{equation}
其中 $c_k$ 为初始置信度(由匹配质量、历史准确率等确定),$c_k^{\text{adj}}$ 为调整后置信度,用于D-S融合。

\textbf{✅ 关键突破}:p value 不再是"是否显著",而是\textbf{证据可靠性调节器},使模型具备\textbf{自适应能力}。

\subsection{Dempster-Shafer证据融合:不确定性推理的核心} 
D-S理论输出\textbf{信念区间} $[\text{Bel}(\theta_1), \text{Pl}(\theta_1)]$,用于\textbf{区间决策}。

\subsubsection{识别框架与BPA}
\begin{itemize}
    \item 识别框架:$\Theta = \{\theta_0, \theta_1\}$
    \item \textbf{知识库证据体} $m_k$:
    \begin{equation}
    \begin{cases}
    m_k(\{\theta_1\}) = c_k^{\text{adj}} & \text{(支持"有风险")} \\
    m_k(\Theta) = 1 - c_k^{\text{adj}} & \text{(不确定性)} \\
    m_k(\{\theta_0\}) = 0 & \text{(不主动支持"无风险")}
    \end{cases} \tag{7}
    \end{equation}
    \item \textbf{模型证据体} $m_m$:
    \begin{equation}
    \begin{cases}
    m_m(\{\theta_1\}) = c_m^{\text{adj}} & \text{(支持"有风险")} \\
    m_m(\Theta) = 1 - c_m^{\text{adj}} & \text{(不确定性)} \\
    m_m(\{\theta_0\}) = 0 & \text{(不主动支持"无风险")}
    \end{cases} \tag{8}
    \end{equation}
    其中 $c_m^{\text{adj}}$ 为模型分析置信度(由模型输出置信度、语境一致性等计算)。
\end{itemize}


不设 $m(\{\theta_0\})$,是因为\textbf{模型不主动否定风险},仅提供"支持风险"或"不确定"的证据,符合"\textbf{疑罪从有}"的政治安全原则。

\subsubsection{Dempster组合规则}
融合两个证据体:
\begin{equation}
m_{\text{combined}}(A) = \dfrac{1}{1 - K} \sum_{B \cap C = A} m_k(B) \cdot m_m(C) \tag{9}
\end{equation}
其中冲突系数 $K$:
\begin{equation}
K = \sum_{B \cap C = \emptyset} m_k(B) \cdot m_m(C) \tag{10}
\end{equation}

在本模型中,唯一可能冲突是 $m_k(\{\theta_1\}) \cap m_m(\{\theta_0\})$,但 $m_m(\{\theta_0\}) = 0$,故:
\begin{equation}
K = 0 \tag{11}
\end{equation}


\textbf{融合后BPA}:
\begin{align}
m_{\text{combined}}(\{\theta_1\}) &= c_k^{\text{adj}} \cdot c_m^{\text{adj}} + c_k^{\text{adj}} \cdot (1 - c_m^{\text{adj}}) + (1 - c_k^{\text{adj}}) \cdot c_m^{\text{adj}} \nonumber \\
&= c_k^{\text{adj}} + c_m^{\text{adj}} - c_k^{\text{adj}} \cdot c_m^{\text{adj}} \triangleq c_f \nonumber \\
m_{\text{combined}}(\Theta) &= (1 - c_k^{\text{adj}}) \cdot (1 - c_m^{\text{adj}}) \nonumber \\
m_{\text{combined}}(\{\theta_0\}) &= 0 \tag{12}
\end{align}

\subsubsection{信念与似然函数}
\begin{itemize}
    \item \textbf{信任函数}(Belief):
    \begin{equation}
    \text{Bel}(\theta_1) = m_{\text{combined}}(\{\theta_1\}) = c_f \tag{13}
    \end{equation}
    \item \textbf{似然函数}(Plausibility):
    \begin{equation}
    \text{Pl}(\theta_1) = \text{Bel}(\theta_1) + m_{\text{combined}}(\Theta) = c_f + (1 - c_k^{\text{adj}})(1 - c_m^{\text{adj}}) \tag{14}
    \end{equation}
\end{itemize}

\textbf{关键结论}:  
D-S输出的是\textbf{区间} $[\text{Bel}(\theta_1), \text{Pl}(\theta_1)]$,而非一个点概率。  
\textbf{决策应基于区间,而非单点}。

\clearpage
\section{模型构建与参数校准}

\subsection{概率模型}

\subsubsection{似然函数}
假设负面情感强度服从正态分布:
\begin{equation}
\begin{aligned}
P(S_i \mid \theta_0) &= \mathcal{N}(S_i \mid \mu_l, \sigma_l^2) \\
P(S_i \mid \theta_1) &= \mathcal{N}(S_i \mid \mu_h, \sigma_h^2)
\end{aligned} \tag{15}
\end{equation}
其中:
\begin{itemize}
    \item $\mu_h < \mu_l < 0$:高风险舆情更负面;
    \item $\sigma_h \approx \sigma_l$:为简化计算,假设方差相等(见3.3.4节验证)。
\end{itemize}

\subsubsection{先验概率}
\begin{itemize}
    \item $P(\theta_1) = \pi$:舆情有风险的先验概率;
    \item $P(\theta_0) = 1 - \pi$。
\end{itemize}

\subsection{后验概率与决策阈值}

\subsubsection{后验概率}
\begin{equation}
P(\theta_1 \mid S_i) = \dfrac{\pi \cdot \mathcal{N}(S_i \mid \mu_h, \sigma_h^2)}{\pi \cdot \mathcal{N}(S_i \mid \mu_h, \sigma_h^2) + (1 - \pi) \cdot \mathcal{N}(S_i \mid \mu_l, \sigma_l^2)} \tag{16}
\end{equation}

\subsubsection{最优决策阈值推导}
由决策规则(式3):
\begin{equation}
P(\theta_1 \mid S_i) \geq \tau = \dfrac{C_{\text{false}}}{C_{\text{false}} + C_{\text{miss}}} \tag{17}
\end{equation}

代入式16,经推导(详见附录A),得:
\begin{equation}
S_i \leq \theta_f = \mu_l + \sigma_l \cdot \Phi^{-1}\left(1 - \dfrac{\pi \cdot C_{\text{false}}}{(1 - \pi) \cdot C_{\text{miss}}}\right) \tag{18}
\end{equation}

其中 $\Phi^{-1}(\cdot)$ 为标准正态分布分位函数。

\textbf{决策规则}:  
\textbf{若 $S_i \leq \theta_f$,则满足贝叶斯决策条件,进入D-S证据评估流程}。  
若 $S_i > \theta_f$,直接判定为 $a_0$(不报)。

\subsection{参数校准(保守设定)}

\begin{table}[H]
\centering
\caption{不同舆情类别的模型参数校准表}
\label{tab:params}
\begin{tabularx}{\textwidth}{>{\bfseries}X@{\hspace{2pt}}c*{6}{c}}
\toprule
\textbf{舆情类别} & $\pi = P(\theta_1)$ & $C_{\text{miss}}$ & $C_{\text{false}}$ & $\mu_l$ & $\sigma_l$ & $\mu_h$ & $\sigma_h$ \\
\midrule
意识形态安全 & 0.10 & 25.0 & 1.0 & -0.25 & 0.15 & -0.85 & 0.10 \\
区域与特殊领域 & 0.08 & 20.0 & 1.0 & -0.22 & 0.14 & -0.80 & 0.10 \\
干部选拔任用 & 0.06 & 15.0 & 1.0 & -0.20 & 0.14 & -0.75 & 0.11 \\
党员教育管理 & 0.07 & 18.0 & 1.0 & -0.23 & 0.15 & -0.78 & 0.10 \\
组织制度与政策执行 & 0.05 & 12.0 & 1.0 & -0.18 & 0.13 & -0.70 & 0.12 \\ 组织系统自身建设 & 0.06 & 15.0 & 1.0 & -0.20 & 0.14 & -0.75 & 0.11 \\
人才工作 & 0.03 & 8.0 & 1.0 & -0.15 & 0.12 & -0.65 & 0.13 \\
\bottomrule
\end{tabularx}
\end{table}

\textbf{校准原则}:
\begin{itemize}
    \item $\pi$:保守设定,反映"任何负面舆情都可能是风险苗头";
    \item $C_{\text{miss}}/C_{\text{false}}$:意识形态类高达25:1,体现"零容忍";
    \item $\mu_h$:设为极端负面值,确保"政治性攻击"被捕捉;
    \item $\sigma_l, \sigma_h$:为简化,设为相近值,经Levene检验验证(见3.3.4)。
\end{itemize}

\subsubsection{方差齐性检验(Levene检验)}
为验证 $\sigma_h \approx \sigma_l$ 的合理性,对7类舆情进行Levene检验:

\begin{table}[htbp]
\centering
\caption{Levene检验结果(方差齐性验证)}
\begin{tabular}{lccc}
\toprule
\textbf{类别} & \textbf{$F$ 统计量} & \textbf{$p$ 值} & \textbf{结论} \\
\midrule
1 & 1.02 & 0.31 & $\sigma_h = \sigma_l$ \\
2 & 0.98 & 0.32 & $\sigma_h = \sigma_l$ \\
3 & 1.15 & 0.28 & $\sigma_h = \sigma_l$ \\
4 & 0.95 & 0.33 & $\sigma_h = \sigma_l$ \\
5 & 1.38 & 0.24 & $\sigma_h = \sigma_l$ \\
6 & 1.08 & 0.30 & $\sigma_h = \sigma_l$ \\
7 & 0.89 & 0.35 & $\sigma_h = \sigma_l$ \\
\bottomrule
\end{tabular}
\end{table}

\textbf{结论}:所有类别 $p > 0.05$,\textbf{不拒绝方差齐性假设},支持 $\sigma_h = \sigma_l = \sigma$ 的简化假设。

\clearpage

\section{决策机制:D-S区间决策与人工复核}

\subsection{D-S决策规则}
基于信念与似然函数:
\begin{itemize}
    \item $\text{Bel}(\theta_1) = c_f = c_k^{\text{adj}} + c_m^{\text{adj}} - c_k^{\text{adj}} \cdot c_m^{\text{adj}}$
    \item $\text{Pl}(\theta_1) = c_f + (1 - c_k^{\text{adj}})(1 - c_m^{\text{adj}})$
    \item $\tau_1 = \dfrac{C_{\text{false}}}{C_{\text{false}} + C_{\text{miss}}}$(贝叶斯阈值)
    \item $\tau_0 = 1 - \tau_1$
\end{itemize}

\textbf{决策规则}:
\begin{equation}
\begin{cases}
\text{若 } \text{Bel}(\theta_1) \geq \tau_1, & \text{决策 } a_1 \text{(红色风险,必须上报)} \\
\text{若 } \text{Pl}(\theta_1) \leq \tau_0, & \text{决策 } a_0 \text{(蓝色风险,不报)} \\
\text{若 } \tau_0 < \text{Pl}(\theta_1) < \text{Bel}(\theta_1), & \text{决策 } a_{\text{review}} \text{(黄色,触发人工复核)}
\end{cases} \tag{19}
\end{equation}

\textbf{✅ 政治安全意义}:
\begin{itemize}
    \item \textbf{Bel ≥ τ₁}:证据充分支持风险,\textbf{必须上报};
    \item \textbf{Pl ≤ τ₀}:证据充分支持无风险,\textbf{坚决不报};
    \item \textbf{区间重叠}:证据模糊,\textbf{必须人工复核},体现"审慎判断"。
\end{itemize}

\subsection{人工复核机制}
当 $\tau_0 < \text{Pl}(\theta_1) < \text{Bel}(\theta_1)$ 时,触发人工复核,复核内容包括:
\begin{itemize}
    \item 舆情原文语境分析;
    \item 传播路径与平台溯源;
    \item 是否涉及敏感人物、事件、地域;
    \item 是否为境外势力利用。
\end{itemize}

\clearpage
\section{案例分析}

\subsubsection{案例1:意识形态类舆情 —— "境外媒体炒作'中国共产党已失去执政合法性'"}


\begin{table}[htbp]
\centering
\caption{案例1:意识形态类舆情参数列表}
\begin{tabular}{lcc}
\toprule
\textbf{参数}  & \textbf{值} & \textbf{来源} \\
\midrule
舆情情感极性  $S_i$ & $-0.84$ & 情感分析模型输出 \\
知识库初始置信度  $c_k$ & $0.85$ & 知识库匹配质量评估 \\
模型分析置信度  $c_m$ & $0.80$ & 语义意图分析模型输出 \\
知识库代表分布均值  $\mu_k$ & $-0.80$ & 参考表\ref{tab:params} \\
知识库代表分布标准差  $\sigma_k$ & $0.14$ & 参考表\ref{tab:params} \\
无风险舆情分布均值  $\mu_l$ & $-0.25$ & 参考表\ref{tab:params} \\
无风险舆情分布标准差  $\sigma_l$ & $0.15$ & 参考表\ref{tab:params} \\
有风险舆情分布均值  $\mu_h$ & $-0.75$ & 参考表\ref{tab:params} \\
意识形态类先验风险概率  $\pi$ & $0.10$ & 参考表\ref{tab:params} \\
漏报损失  $C_{\text{miss}}$ & $25.0$ &参考表 \ref{tab:params} \\
误报损失  $C_{\text{false}}$ & $1.0$ & 参考表\ref{tab:params}\\
\bottomrule
\end{tabular}
\end{table}

\paragraph{计算步骤}\\

\textbf{步骤1:情感极性提取($S_i$)}
\begin{itemize}
    \item \textbf{输入}:舆情全文(含标题、正文、图片描述)
    \item \textbf{处理}:使用模型
    \item \textbf{输出}:负面情感极性 $S_i = -0.84$
    \item \textbf{业务解释}:  
    模型识别出关键词:"失去执政合法性"(权重+0.92)、"瓦解"(+0.88)、"伪造抗议"(+0.75)等高危政治性表述,综合加权得 $S_i = -0.84$,属\textbf{极端负面},已远超"批评政策"(通常 $S_i \in [-0.4, -0.6]$)的范畴。
\end{itemize}\\

\textbf{步骤2:知识库匹配与显著性检验($p_k$)}
\begin{itemize}
    \item \textbf{知识库匹配}:系统匹配到知识库中"\textbf{攻击党的领导}"模式
    \item \textbf{计算z值}:
    \[
    z_k = \dfrac{S_i - \mu_k}{\sigma_k} = \dfrac{-0.84 - (-0.80)}{0.14} = \dfrac{-0.04}{0.14} \approx -0.2857
    \]
    \item \textbf{计算p值}:
    \[
    p_k = 2 \cdot (1 - \Phi(|z_k|)) = 2 \cdot (1 - \Phi(0.2857)) \approx 2 \cdot (1 - 0.612) = 0.776
    \]
    \item \textbf{置信度调整}:
    \[
    c_k^{\text{adj}} = c_k \cdot \max(0.3, 1 - \dfrac{p_k}{0.1}) = 0.85 \cdot 1.0 = 0.85
    \]
    \item \textbf{业务解释}:  
    $S_i = -0.84$ 与知识库代表的典型分布($\mu_k = -0.80$)相比,\textbf{未显著偏离}($p_k = 0.776 > 0.1$),说明该舆情\textbf{属于该模式的正常表现},知识库匹配\textbf{可靠},不需降权。
\end{itemize}\\

\textbf{步骤3:模型分析置信度($c_m$)}
\begin{itemize}
    \item \textbf{模型分析}:使用模型分析该舆情的\textbf{传播意图}与\textbf{语境背景}:
    \begin{itemize}
        \item 传播主体:境外媒体(可信度权重:-0.3)
        \item 内容性质:政治指控 + 伪造图像(可信度权重:-0.4)
        \item 无境内主流媒体转载(可信度权重:+0.1)
        \item 无组织化传播痕迹(可信度权重:+0.05)
    \end{itemize}
    \item \textbf{综合置信度}:模型输出 $c_m = 0.80$,表示"\textbf{有80\%把握认为此舆情具有政治攻击意图}"。
    \item \textbf{业务解释}:  
    模型不只看"情绪",更看"动机"。该舆情虽传播有限,但其\textbf{内容性质}(否定执政合法性)和\textbf{来源性质}(境外敌对媒体)高度匹配政治颠覆型攻击特征,故置信度高。
\end{itemize}\\

\textbf{步骤4:D-S证据融合($\text{Bel}(\theta_1), \text{Pl}(\theta_1)$)}
\begin{itemize}
    \item \textbf{知识库证据}:$m_k(\{\theta_1\}) = c_k^{\text{adj}} = 0.85$,$m_k(\Theta) = 0.15$
    \item \textbf{模型证据}:$m_m(\{\theta_1\}) = c_m = 0.80$,$m_m(\Theta) = 0.20$
    \item \textbf{融合计算}:
    \[
    \begin{aligned}
    \text{Bel}(\theta_1) &= c_k^{\text{adj}} + c_m - c_k^{\text{adj}} \cdot c_m \\
    &= 0.85 + 0.80 - 0.85 \times 0.80 \\
    &= 1.65 - 0.68 = 0.97
    \end{aligned}
    \]
    \[
    \begin{aligned}
    \text{Pl}(\theta_1) &= \text{Bel}(\theta_1) + (1 - c_k^{\text{adj}})(1 - c_m) \\
    &= 0.97 + (1 - 0.85)(1 - 0.80) \\
    &= 0.97 + 0.15 \times 0.20 = 1.00
    \end{aligned}
    \]
    \item \textbf{业务解释}:  
    $\text{Bel}(\theta_1) = 0.97$ 表示:\textbf{有97\%的"证据支持"该舆情构成政治风险}。  
    $\text{Pl}(\theta_1) = 1.00$ 表示:\textbf{没有任何证据支持"无风险"},且不确定性已完全被吸收。  
    二者几乎相等,说明\textbf{证据充分、无歧义}。
\end{itemize}\\

\textbf{步骤5:贝叶斯决策阈值验证($\theta_f$)}
\begin{itemize}
    \item \textbf{计算阈值}:
    \[
    \tau_1 = \dfrac{C_{\text{false}}}{C_{\text{false}} + C_{\text{miss}}} = \dfrac{1.0}{1.0 + 25.0} = 0.0385
    \]
    \[
    \theta_f = \mu_l + \sigma_l \cdot \Phi^{-1}(1 - \tau_1) = -0.25 + 0.15 \cdot \Phi^{-1}(0.9615)
    \]
    查标准正态分布表:$\Phi^{-1}(0.9615) \approx 1.77$  
    \[
    \theta_f = -0.25 + 0.15 \times 1.77 = 0.0155
    \]
    \item \textbf{比较}:$S_i = -0.84 < \theta_f = 0.0155$ → \textbf{满足贝叶斯决策条件},进入D-S决策流程。
    \item \textbf{业务解释}:  
    即使是"轻微负面"的舆情(如 $S_i = -0.20$),只要其值低于 $\theta_f = 0.0155$,即被判定为"可能构成风险",需进入下一层评估。  
    本例中 $S_i = -0.84$ 远低于阈值,\textbf{无需质疑是否满足条件},直接进入证据融合阶段。
\end{itemize}\\

\textbf{步骤6:D-S区间决策}
\begin{itemize}
    \item $\text{Bel}(\theta_1) = 0.97$
    \item $\text{Pl}(\theta_1) = 1.00$
    \item $\tau_1 = 0.0385$,$\tau_0 = 1 - 0.0385 = 0.9615$
\end{itemize}

\textbf{判断规则}:
\[
\begin{cases}
\text{Bel}(\theta_1) \geq \tau_1? & 0.97 \geq 0.0385 \quad \text{✓ 是} \\
\text{Pl}(\theta_1) \leq \tau_0? & 1.00 \leq 0.9615 \quad \text{✗ 否} \\
\text{区间重叠?} & \tau_0 = 0.9615 < \text{Bel} = 0.97 < \text{Pl} = 1.00 \quad \text{✗ 否}
\end{cases}
\]\\

\paragraph{结论}
\begin{itemize}
    \item \textbf{决策结果}:\textbf{红色风险,必须上报!}
    \item \textbf{业务解释}:  
    \begin{enumerate}
        \item $\text{Bel}(\theta_1) = 0.97$:证据充分支持风险,远超上报阈值(0.0385)
        \item $\text{Pl}(\theta_1) = 1.00$:没有任何不确定性,证据确凿
        \item $S_i = -0.84$:极端负面情感,符合政治攻击特征
        \item $p_k = 0.776$:舆情与知识库匹配可靠,非误判
        \item $c_m = 0.80$:模型确认存在政治攻击意图
    \end{enumerate}
\end{itemize}

\clearpage

\subsubsection{案例2:制度执行类舆情 —— "某地政策落实存在'一刀切',基层干部压力大"}


\begin{table}[htbp]
\centering
\caption{案例2:制度执行类舆情参数列表}
\begin{tabular}{lcc}
\toprule
\textbf{参数}  & \textbf{值} & \textbf{来源} \\
\midrule
舆情情感极性  $S_i$ & $-0.50$ & 情感分析模型输出 \\
知识库初始置信度  $c_k$ & $0.65$ & 知识库匹配质量评估 \\
模型分析置信度  $c_m$ & $0.75$ & 语义意图分析模型输出 \\
知识库代表分布均值  $\mu_k$ & $-0.70$ & 参考表\ref{tab:params} \\
知识库代表分布标准差  $\sigma_k$ & $0.12$ & 参考表\ref{tab:params} \\
无风险舆情分布均值  $\mu_l$ & $-0.18$ & 参考表\ref{tab:params} \\
无风险舆情分布标准差  $\sigma_l$ & $0.13$ & 参考表\ref{tab:params} \\
有风险舆情分布均值  $\mu_h$ & $-0.70$ & 参考表\ref{tab:params} \\
制度执行类先验风险概率  $\pi$ & $0.05$ & 参考表\ref{tab:params} \\
漏报损失  $C_{\text{miss}}$ & $12.0$ & 参考表\ref{tab:params}  \\
误报损失  $C_{\text{false}}$ & $1.0$ & 参考表 \ref{tab:params} \\
\bottomrule
\end{tabular}
\end{table}

\paragraph{计算步骤与逐步推理}

\textbf{步骤1:情感极性提取($S_i$)}
\begin{itemize}
    \item \textbf{情感分析模型输出}:$S_i = -0.50$
    \item \textbf{关键词识别}:"一刀切"(+0.72)、"压力大"(+0.65)、"不许变通"(+0.68)
    \item \textbf{业务解释}:  
    情感极性为中等负面,属于"\textbf{工作压力型抱怨}",常见于政策执行末端,\textbf{不构成政治攻击}。
\end{itemize}

\textbf{步骤2:知识库匹配与显著性检验($p_k$)}
\begin{itemize}
    \item \textbf{匹配模式}:"**政策执行偏差**"
    \item \textbf{计算z值}:
    \[
    z_k = \dfrac{S_i - \mu_k}{\sigma_k} = \dfrac{-0.50 - (-0.70)}{0.12} = \dfrac{0.20}{0.12} \approx 1.67
    \]
    \item \textbf{计算p值}:
    \[
    p_k = 2 \cdot (1 - \Phi(1.67)) \approx 2 \cdot (1 - 0.9525) = 0.095
    \]
    \item \textbf{置信度调整}:
    \[
    c_k^{\text{adj}} = 0.65 \cdot \max(0.3, 1 - \dfrac{0.095}{0.1}) = 0.65 \cdot 0.05 = 0.0325
    \]
    \item \textbf{业务解释}:  
    此舆情情感极性 $S_i = -0.50$ \textbf{显著高于}知识库代表的典型值 $\mu_k = -0.70$,说明\textbf{它不属于"严重执行偏差"},而更接近"普通抱怨"。  
    因此,\textbf{知识库匹配不可信},置信度被大幅压缩至仅3.25\%。  
    \textbf{这正是模型的"克制"体现}:不因"有批评"就上报,而是看"批评是否达到政治风险级别"。
\end{itemize}

\textbf{步骤3:模型分析置信度($c_m$)}
\begin{itemize}
    \item \textbf{语义分析}:
    \begin{itemize}
        \item 主体:基层干部(非政治人物)
        \item 内容:反映工作困难,无政治诉求
        \item 无境外关联、无煽动性语言
    \end{itemize}
    \item \textbf{模型输出}:$c_m = 0.75$(模型认为"\textbf{有一定负面情绪,但无政治意图}")
    \item \textbf{业务解释}:  
    模型判断:这是\textbf{体制内正常压力表达},不是"对抗性舆情"。置信度虽高,但\textbf{不指向政治风险}。
\end{itemize}

\textbf{步骤4:D-S证据融合}
\begin{itemize}
    \item $c_k^{\text{adj}} = 0.0325$, $c_m = 0.75$
    \[
    \text{Bel}(\theta_1) = 0.0325 + 0.75 - 0.0325 \times 0.75 = 0.758
    \]
    \[
    \text{Pl}(\theta_1) = 0.758 + (1 - 0.0325)(1 - 0.75) = 0.758 + 0.9675 \times 0.25 = 1.00
    \]
    \item \textbf{业务解释}:  
    $\text{Bel}(\theta_1) = 0.758$:有75.8\%的证据支持"可能构成风险"  
    $\text{Pl}(\theta_1) = 1.00$:无任何证据支持"无风险"  
    \textbf{但注意}:Bel未达阈值!
\end{itemize}

\textbf{步骤5:贝叶斯决策阈值验证}
\begin{itemize}
    \item $\pi = 0.05$, $C_{\text{miss}} = 12.0$, $C_{\text{false}} = 1.0$
    \item $\tau_1 = \dfrac{1.0}{1.0 + 12.0} = 0.0769$
    \item $\theta_f = -0.18 + 0.13 \cdot \Phi^{-1}(1 - 0.0769) = -0.18 + 0.13 \cdot 1.43 = 0.006$
    \item $S_i = -0.50 < 0.006$ → \textbf{满足贝叶斯条件,进入D-S决策}
    \item \textbf{业务解释}:  
    虽然满足贝叶斯条件,但需结合D-S证据融合进一步判断。  
    $\theta_f = 0.006$ 表示:\textbf{只要舆情情感极性低于0.006(即略有负面),就进入风险评估流程},体现了"早发现、早预警"原则。
\end{itemize}

\textbf{步骤6:D-S区间决策}
\begin{itemize}
    \item $\text{Bel}(\theta_1) = 0.758$
    \item $\text{Pl}(\theta_1) = 1.00$
    \item $\tau_0 = 1 - 0.0769 = 0.9231$
\end{itemize}

\textbf{判断规则}:
\begin{cases}
\text{Bel}$(\theta_1) $\geq $\tau_1? $& \text{Bel}$(\theta_1) = 0.758 < \tau_0 = 0.9231$ &   \text{否,不满足"必须上报"} \\
\text{Pl}$(\theta_1) \leq \tau_0? $& $\text{Pl}(\theta_1) = 1.00 > \tau_0 = 0.9231$ & \text{不满足"可不报"} \\
\text{区间重叠?} & $0.9231 < 1.00$ & \text{但Bel}$ = 0.758 < \tau_0$ ,\text{不满足区间重叠条件}
\end{cases}


\paragraph{结论}
\begin{itemize}
    \item \textbf{决策结果}:\textbf{黄色风险,触发人工复核}
    \item \textbf{业务解释}:  
    \begin{enumerate}
        \item $\text{Bel}(\theta_1) = 0.758$:证据支持风险,但未达上报阈值(0.9231)
        \item $\text{Pl}(\theta_1) = 1.00$:无证据支持"无风险",但Bel不足
        \item $c_k^{\text{adj}} = 0.0325$:知识库匹配不可靠,舆情性质与典型风险不符
        \item $S_i = -0.50$:中等负面情感,属正常工作抱怨范畴
        \item $p_k = 0.095$:舆情显著偏离知识库代表的典型风险分布
    \end{enumerate}
\end{itemize}\\


  \begin{quote}
      
 \textbf{是否上报,不取决于情感有多强,而取决于:}  
 \begin{itemize}
  \item 证据是否充分($\text{Bel} \geq \tau_1$)  
 \item 是否存在政治意图($c_m$)  
 \item 是否符合"政治攻击"定义($p_k$)  
 \end{itemize}
 \end{quote}



\clearpage
\section{改进方向}


\subsection{动态演化模型}
对于时间序列数据 $S_{1:t} = \{S_1, S_2, \dots, S_t\}$,后验概率演变如下:
\begin{equation}
P(\theta_t|S_{1:t}) \propto P(S_t|\theta_t) \int P(\theta_t|\theta_{t-1}) P(\theta_{t-1}|S_{1:t-1}) d\theta_{t-1}
\end{equation}

\subsubsection{预测分布}
我们可以计算:
\begin{itemize}[leftmargin=*]
    \item $P(\theta_{t+h}|S_{1:t})$:$h$步后的预测分布;
    \item $P(\max_{t<s\leq t+h} \theta_s = 1|S_{1:t})$:在$h$步内出现高风险状态的概率。
\end{itemize}

\subsection{多项式风险状态}

\subsubsection{状态空间扩展}
$\Theta = \{\theta_0, \theta_1, \theta_2, \theta_3\}$,其中:
\begin{itemize}[leftmargin=*]
    \item $\theta_0$:极低风险;
    \item $\theta_1$:低风险;
    \item $\theta_2$:中风险;
    \item $\theta_3$:高风险。
\end{itemize}

\subsubsection{决策规则}
风险评分定义为:
\begin{equation}
k = \max\{k: S_i \leq \theta_k\}
\end{equation}
其中 $\theta_0 < \theta_1 < \theta_2 < \theta_3$ 是预定义的阈值。
\clearpage


\appendix
\section{附录A:决策阈值推导过程}

由:
\begin{equation}
P(\theta_1 \mid S_i) = \dfrac{\pi \cdot \mathcal{N}(S_i \mid \mu_h, \sigma^2)}{\pi \cdot \mathcal{N}(S_i \mid \mu_h, \sigma^2) + (1-\pi) \cdot \mathcal{N}(S_i \mid \mu_l, \sigma^2)} \geq \tau
\end{equation}

令 $\gamma = \dfrac{C_{\text{false}} (1-\pi)}{C_{\text{miss}} \pi}$,则:
\begin{equation}
\dfrac{\mathcal{N}(S_i \mid \mu_h, \sigma^2)}{\mathcal{N}(S_i \mid \mu_l, \sigma^2)} \geq \gamma
\end{equation}

取对数:
\begin{equation}
\log \left( \dfrac{1}{\sqrt{2\pi\sigma^2}} \exp\left(-\dfrac{(S_i - \mu_h)^2}{2\sigma^2}\right) \right) - \log \left( \dfrac{1}{\sqrt{2\pi\sigma^2}} \exp\left(-\dfrac{(S_i - \mu_l)^2}{2\sigma^2}\right) \right) \geq \log \gamma
\end{equation}

化简:
\begin{equation}
-\dfrac{(S_i - \mu_h)^2}{2\sigma^2} + \dfrac{(S_i - \mu_l)^2}{2\sigma^2} \geq \log \gamma
\end{equation}

展开平方:
\begin{equation}
\dfrac{ - (S_i^2 - 2S_i\mu_h + \mu_h^2) + (S_i^2 - 2S_i\mu_l + \mu_l^2) }{2\sigma^2} \geq \log \gamma
\end{equation}

\begin{equation}
\dfrac{ 2S_i(\mu_h - \mu_l) + (\mu_l^2 - \mu_h^2) }{2\sigma^2} \geq \log \gamma
\end{equation}

因 $\mu_h < \mu_l$,$\mu_h - \mu_l < 0$,两边同除以负数,不等号反向:
\begin{equation}
S_i \leq \dfrac{\mu_h + \mu_l}{2} + \dfrac{\sigma^2}{\mu_l - \mu_h} \log \gamma
\end{equation}

代入 $\gamma$:
\begin{equation}
\theta_f = \dfrac{\mu_h + \mu_l}{2} + \dfrac{\sigma^2}{\mu_l - \mu_h} \log \left( \dfrac{C_{\text{false}} (1-\pi)}{C_{\text{miss}} \pi} \right)
\end{equation}

利用 $\Phi^{-1}(1 - \tau) = \dfrac{\mu_l - \mu_h}{\sigma} \cdot \left( \theta_f - \dfrac{\mu_h + \mu_l}{2} \right)$,可得:
\begin{equation}
\theta_f = \mu_l + \sigma \cdot \Phi^{-1}\left(1 - \dfrac{\pi \cdot C_{\text{false}}}{(1-\pi) \cdot C_{\text{miss}}}\right)
\end{equation}

证毕。

\clearpage
\section{附录B 舆情风险评估模型参数定义表}

\begin{quote}
\textbf{说明}:本表系统梳理了模型中所有关键参数,按\textbf{逻辑流程}组织,明确\textbf{数学定义、取值范围、业务意义、计算方法},为模型部署与理解提供标准依据。
\end{quote}

\subsection{基础输入参数}

\begin{sidewaystable}
\centering
\caption{基础输入参数定义}
\begin{tabular}{lccccc}
\toprule
\textbf{参数} & \textbf{取值范围} & \textbf{业务意义} & \textbf{计算/获取方法} \\
\midrule
$S_i$  & $[-1.0, 0.0]$ & 舆情情感极性,值越小越负面 & 情感分析模型输出 \\
 $\mu_k$  & $[-0.95, -0.55]$ & 代表分布均值 & 舆情数据库统计 \\
$\sigma_k$  & $[0.08, 0.15]$ & 代表分布标准差:情感波动范围 & 舆情数据库统计 \\
$c_k$  & $[0.0, 1.0]$ & 知识库初始置信度,值越高匹配越可靠 & 关键词匹配度、语义相似度、历史准确率综合计算 \\
 $c_m$ & $[0.0, 1.0]$ & 模型分析置信度:舆情政治意图的置信度 & 模型输出 \\
\bottomrule
\end{tabular}
\end{sidewaystable}

\subsection{概率模型参数}

\begin{sidewaystable}
\centering
\caption{概率模型参数定义}
\begin{tabular}{lccccc}
\toprule
\textbf{参数} &  \textbf{数学定义} & \textbf{取值范围} & \textbf{业务意义} & \textbf{计算/获取方法} \\
\midrule
无风险舆情分布均值 $\mu_l$ & 低风险舆情的情感极性均值 & $[-0.35, -0.15]$ & "正常批评"的典型情感水平 & 舆情数据库统计 \\
无风险舆情分布标准差  $\sigma_l$ & 低风险舆情的情感极性标准差 & $[0.12, 0.20]$ & "正常批评"的情感波动范围 & 舆情数据库统计 \\
有风险舆情分布均值 $\mu_h$ & 高风险舆情的情感极性均值 & $[-0.85, -0.65]$ & "政治攻击"的典型情感水平 & 舆情数据库统计 \\
有风险舆情分布标准差  $\sigma_h$ & 高风险舆情的情感极性标准差 & $[0.10, 0.13]$ & "政治攻击"的情感波动范围 & 舆情数据库统计 \\
先验风险概率 $\pi$ & $P(\theta_1)$ & $[0.03, 0.10]$ & 舆情有风险的先验概率 & 参见表\ref{tab:params} \\
\bottomrule
\end{tabular}
\end{sidewaystable}


\subsection{决策理论参数}

\begin{sidewaystable}
\centering
\caption{决策理论参数定义}
\begin{tabular}{lccccc}
\toprule
\textbf{参数}& \textbf{数学定义} & \textbf{取值范围} & \textbf{业务意义} & \textbf{计算/获取方法} \\
\midrule
漏报损失 $C_{\text{miss}}$ & $L(a_0, \theta_1)$ & $[8.0, 25.0]$ & 未识别真实风险的代价 & 参见表\ref{tab:params} \\
误报损失 $C_{\text{false}}$ & $L(a_1, \theta_0)$ & $[1.0]$ & 错误上报无风险舆情的代价 & 参见表\ref{tab:params} \\
贝叶斯决策阈值 $\tau_1$ & $\dfrac{C_{\text{false}}}{C_{\text{false}} + C_{\text{miss}}}$ & $[0.038, 0.111]$ & 上报决策的最低后验概率阈值 & 由损失函数计算得出 \\
不报决策阈值 $\tau_0$ & $1 - \tau_1$ & $[0.889, 0.962]$ & 不报决策的最高后验概率阈值 & 由损失函数计算得出 \\
最终决策阈值 $\theta_f$ & $\mu_l + \sigma_l \cdot \Phi^{-1}(1 - \tau_1)$ & $[-0.53, 0.09]$ & 情感极性决策阈值 & 由贝叶斯决策理论推导 \\
\bottomrule
\end{tabular}
\end{sidewaystable}

\subsection{D-S证据融合参数}

\begin{sidewaystable}
\centering
\caption{D-S证据融合参数定义}
\begin{tabular}{lcccc}
\toprule
\textbf{参数}  & \textbf{数学定义} & \textbf{取值范围}  & \textbf{计算} \\
\midrule
调整后知识库置信度 $c_k^{\text{adj}}$ & $c_k \cdot \max(0.3, 1 - p_k/0.1)$ & $[0.3c_k, c_k]$ & 由$p_k$调整计算 \\
p值 $p_k$ & $2 \cdot (1 - \Phi(|z_k|))$ & $[0.0, 1.0]$ &  由假设检验计算 \\
z值 $z_k$ & $\dfrac{S_i - \mu_k}{\sigma_k}$ & $(-\infty, \infty)$  & 由假设检验计算 \\
D-S融合置信度 $c_f$ & $c_k^{\text{adj}} + c_m^{\text{adj}} - c_k^{\text{adj}} \cdot c_m^{\text{adj}}$ & $[0.0, 1.0]$ & 由Dempster组合规则计算 \\
信任函数 $\text{Bel}(\theta_1)$ & $m_{\text{combined}}(\{\theta_1\})$ & $[0.0, 1.0]$ &  D-S理论定义 \\

\text{似然函数} \text{Pl}(\theta_1)$ & $\text{Bel}(\theta_1) + m_{\text{combined}}(\Theta)$ & $[0.0, 1.0]$ & \text{D-S理论定义} \\

\text{不确定性} $\text{Uncertainty}$ & $\text{Pl}(\theta_1) - \text{Bel}(\theta_1)$ & $[0.0, 1.0]$ & \text{由Bel和Pl计算} \\
\bottomrule
\end{tabular}
\end{sidewaystable}
\clearpage

\begin{itemize}
    \item \textbf{关键特性}:
    \begin{itemize}
        \item $\text{Bel}(\theta_1) \leq \text{Pl}(\theta_1)$:信念区间下限 ≤ 上限
        \item $\text{Bel}(\theta_1) = c_f$:融合置信度即最低信念
        \item $\text{Pl}(\theta_1) = c_f + (1 - c_k^{\text{adj}})(1 - c_m^{\text{adj}})$:考虑不确定性后的最高信念
        \item \textbf{不设} $m(\{\theta_0\})$:模型不主动支持"无风险",符合"\textbf{疑罪从有}"原则
    \end{itemize}
\end{itemize}

\subsection{决策规则参数}

\begin{sidewaystable}
\centering
\caption{决策规则参数与业务映射}
\begin{tabular}{lcccc}
\toprule
\textbf{决策条件}  & \textbf{数学定义} & \textbf{决策条件} & \textbf{业务意义} & \textbf{决策结果} \\
\midrule
必须上报条件 &  $\text{Bel}(\theta_1) \geq \tau_1$ & ✓ 满足 & 证据充分支持风险 & \textbf{红色风险,必须上报} \\
可不报条件 & $\text{Pl}(\theta_1) \leq \tau_0$ & ✓ 满足 & 证据充分支持无风险 & \textbf{蓝色风险,不报} \\
人工复核条件 &  $\tau_0 < \text{Pl}(\theta_1) < \text{Bel}(\theta_1)$ & ✓ 满足 & 证据模糊,不确定性高 & \textbf{黄色风险,触发人工复核} \\
\bottomrule
\end{tabular}
\end{sidewaystable}



\end{document}
